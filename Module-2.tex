\section{MODULE II.}
\subsection{Gas giant planets.}

Origin and evolution. Internal structures, chemical compositions, atmospheres, magnetic fields, physical conditions, surface morphologies, rings.
Complementary information: chapters 12 and 14 (except moons).

You should be familiar with:

\begin{itemize}
\item Layout of the solar system by regions, types of bodies, relative sizes and distances in astronomical units.
\item Common characteristics of gas giant planets in regards of compositions, movements, presence of rings and moons.
\item Morphologies of the visible surfaces (top layers of atmospheres).
\item Internal structures, chemical compositions, atmospheric pressures and temperatures, magnetic fields.
\end{itemize}

PHAS 1516 Module II Gas Giant planets. Reference material:

\begin{itemize}
\item Universe textbook. Chapters: 7, 8, 12 and 14 (Uranus and Neptune only for the moment).
\item Wikipedia entries for the Galileo mission to Jupiter and the Cassini mission to Saturn.
\item \url{http://hubblesite.org/newscenter/archive/releases/solar-system/jupiter/}
\item \url{http://hubblesite.org/newscenter/archive/releases/solar-system/saturn/}
\item \url{https://commons.wikimedia.org/wiki/Poles_of_Saturn}
\item \url{https://en.wikipedia.org/wiki/Saturn's_hexagon}
\item \url{http://hubblesite.org/newscenter/archive/releases/solar-system/uranus/}
\item \url{http://hubblesite.org/newscenter/archive/releases/solar-system/neptune/}
\end{itemize}