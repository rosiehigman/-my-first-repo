\section{MODULE I - Introduction.}

Formation of stars and solar systems in the context of the cosmic timeline 13.8 billion years old. Special version of the H-R diagram. Atomic structure. Periodic table. Isotopes. Stability in atomic nuclei. Radioactivity. Radio active decay and its applications in dating long periods of time including the age of the solar system. Scale model of the solar system. General layout according to main regions in terms of astronomical units.

Complementary information: Universe chapters 8 and 7, in this order (comparative planetology).

You should be familiar with:

\begin{itemize}
\item Location of the formation of solar systems and our own in particular along the cosmic timeline of nearly 14 billion years
\item Atomic structure, the periodic table and radioactive isotopes
\item Basic principles of radioactive dating
\item Relevance of the initial mass in the nature and life of stars, planets and smaller bodies
\item Process of formation of the solar system
\item Scale model of the solar system
\item Diverse regions of the solar system according to the types of bodies that populate them
\item \url{http://en.wikipedia.org/wiki/Radiometric_dating}
\item \url{http://en.wikipedia.or g/wiki/Radiocarbon_dating}
\item \url{https://en.wikipedia.org/wiki/Atom}
\end{itemize}