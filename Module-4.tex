MODULE IV.- The Moon. Origin. Internal structure, chemical composition. Surface morphology. Historical relevance.

\begin{itemize}
\item The  current view on the formation of the Earth-Moon system
\item The  historical relevance of the Moon as the only celestial body that can be properly seen with the naked eye
\item The  different kinds of lunar terrains
\item The  process of cratering along history. Morphology, density, sizes, etc
\item The  internal structure of the Moon and its asymmetric
crust related to the surface markings, seas and gravitational locking
\item The techniques used to map the inner structure of the Moon
\end{itemize}

\begin{itemize}
\item Universe, Chapter 10
\item Lunar core: \url{http://www.nasa.gov/topics/moonmars/features/lunar_core.html}
\item Lunar rock samples: \url{http://curator.jsc.nasa.gov/Lunar/index.cfm}
\item Lunar geology: \url{https://en.wikipedia.org/wiki/Geology_of_the_Moon}
\item Lunar citizen science: \url{http://www.moonzoo.org/Lunar_Surface}
\item Water on the Moon: \url{https://en.wikipedia.org/wiki/Lunar_water}
\item GRAIL \url{https://en.wikipedia.org/wiki/Gravity_Recovery_and_Interior_Laboratory}
\item Lunar missions: \\
\url{https://en.wikipedia.org/wiki/List_of_missions_to_the_Moon} \\
Apollo 15: \url{https://www.youtube.com/watch?v=6xTGzesCsjs} \\
Apollo 16: \url{https://www.youtube.com/watch?v=-xc61kv4aH0} \\
Apollo 17: \url{https://www.youtube.com/watch?v=mGl0EoEo38U}
\end{itemize}