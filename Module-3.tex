\section{MODULE III.}
\subsection{Terrestrial bodies.}
Planets and moons (except the Earth and the Moon). Internal structures, chemical compositions, atmospheres, magnetic fields, physical conditions, surface morphologies. Gravitational locking.
Complementary information: chapters 11, 13 and 14 (moons only).

At the end of this module you should be able to describe the following aspects of the major terrestrial bodies of the solar system: (Note: Earth and Moon excluded)

\begin{itemize}
\item Processes of formation, internal structures including layers, densities and general compositions
\item Surfaces: morphology and composition, evaluate their ages and identify the processes that change them
\item Atmospheres of the relevant ones, including composition, layers and physical properties
\item Highlight at least one characteristic that make some of
these bodies unique (i.e. Io’s volcanos or Titan’s atmosphere)
\end{itemize}

\begin{itemize}
\item Universe, Freedman. Chapters 11, 13 and 14
\item Wikipedia for each of the bodies mentioned here
\item \url{http://messenger.jhuapl.edu/}
\item \url{http://www2.jpl.nasa.gov/magellan/}
\item \url{http://en.wikipedia.org/wiki/Exploration_of_Mars}
\item \url{http://science.nasa.gov/missions/galileo/}
\item \url{http://saturn.jpl.nasa.gov/science/moons/enceladus/}
\item \url{https://en.wikipedia.org/wiki/Titan_(moon)}
\item \url{http://www.nasa.gov/mission_pages/cassini/main/#.VQfdQ47pzkY}
\item \url{http://voyager.jpl.nasa.gov/gallery/uranus.html}
\item \url{http://voyager.jpl.nasa.gov/gallery/neptune.html}
\item \url{https://en.wikipedia.org/wiki/Pluto}
\end{itemize}